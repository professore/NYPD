% Options for packages loaded elsewhere
\PassOptionsToPackage{unicode}{hyperref}
\PassOptionsToPackage{hyphens}{url}
%
\documentclass[
]{article}
\usepackage{amsmath,amssymb}
\usepackage{lmodern}
\usepackage{ifxetex,ifluatex}
\ifnum 0\ifxetex 1\fi\ifluatex 1\fi=0 % if pdftex
  \usepackage[T1]{fontenc}
  \usepackage[utf8]{inputenc}
  \usepackage{textcomp} % provide euro and other symbols
\else % if luatex or xetex
  \usepackage{unicode-math}
  \defaultfontfeatures{Scale=MatchLowercase}
  \defaultfontfeatures[\rmfamily]{Ligatures=TeX,Scale=1}
\fi
% Use upquote if available, for straight quotes in verbatim environments
\IfFileExists{upquote.sty}{\usepackage{upquote}}{}
\IfFileExists{microtype.sty}{% use microtype if available
  \usepackage[]{microtype}
  \UseMicrotypeSet[protrusion]{basicmath} % disable protrusion for tt fonts
}{}
\makeatletter
\@ifundefined{KOMAClassName}{% if non-KOMA class
  \IfFileExists{parskip.sty}{%
    \usepackage{parskip}
  }{% else
    \setlength{\parindent}{0pt}
    \setlength{\parskip}{6pt plus 2pt minus 1pt}}
}{% if KOMA class
  \KOMAoptions{parskip=half}}
\makeatother
\usepackage{xcolor}
\IfFileExists{xurl.sty}{\usepackage{xurl}}{} % add URL line breaks if available
\IfFileExists{bookmark.sty}{\usepackage{bookmark}}{\usepackage{hyperref}}
\hypersetup{
  pdftitle={NYPD},
  pdfauthor={J. Baldrich},
  hidelinks,
  pdfcreator={LaTeX via pandoc}}
\urlstyle{same} % disable monospaced font for URLs
\usepackage[margin=1in]{geometry}
\usepackage{color}
\usepackage{fancyvrb}
\newcommand{\VerbBar}{|}
\newcommand{\VERB}{\Verb[commandchars=\\\{\}]}
\DefineVerbatimEnvironment{Highlighting}{Verbatim}{commandchars=\\\{\}}
% Add ',fontsize=\small' for more characters per line
\usepackage{framed}
\definecolor{shadecolor}{RGB}{248,248,248}
\newenvironment{Shaded}{\begin{snugshade}}{\end{snugshade}}
\newcommand{\AlertTok}[1]{\textcolor[rgb]{0.94,0.16,0.16}{#1}}
\newcommand{\AnnotationTok}[1]{\textcolor[rgb]{0.56,0.35,0.01}{\textbf{\textit{#1}}}}
\newcommand{\AttributeTok}[1]{\textcolor[rgb]{0.77,0.63,0.00}{#1}}
\newcommand{\BaseNTok}[1]{\textcolor[rgb]{0.00,0.00,0.81}{#1}}
\newcommand{\BuiltInTok}[1]{#1}
\newcommand{\CharTok}[1]{\textcolor[rgb]{0.31,0.60,0.02}{#1}}
\newcommand{\CommentTok}[1]{\textcolor[rgb]{0.56,0.35,0.01}{\textit{#1}}}
\newcommand{\CommentVarTok}[1]{\textcolor[rgb]{0.56,0.35,0.01}{\textbf{\textit{#1}}}}
\newcommand{\ConstantTok}[1]{\textcolor[rgb]{0.00,0.00,0.00}{#1}}
\newcommand{\ControlFlowTok}[1]{\textcolor[rgb]{0.13,0.29,0.53}{\textbf{#1}}}
\newcommand{\DataTypeTok}[1]{\textcolor[rgb]{0.13,0.29,0.53}{#1}}
\newcommand{\DecValTok}[1]{\textcolor[rgb]{0.00,0.00,0.81}{#1}}
\newcommand{\DocumentationTok}[1]{\textcolor[rgb]{0.56,0.35,0.01}{\textbf{\textit{#1}}}}
\newcommand{\ErrorTok}[1]{\textcolor[rgb]{0.64,0.00,0.00}{\textbf{#1}}}
\newcommand{\ExtensionTok}[1]{#1}
\newcommand{\FloatTok}[1]{\textcolor[rgb]{0.00,0.00,0.81}{#1}}
\newcommand{\FunctionTok}[1]{\textcolor[rgb]{0.00,0.00,0.00}{#1}}
\newcommand{\ImportTok}[1]{#1}
\newcommand{\InformationTok}[1]{\textcolor[rgb]{0.56,0.35,0.01}{\textbf{\textit{#1}}}}
\newcommand{\KeywordTok}[1]{\textcolor[rgb]{0.13,0.29,0.53}{\textbf{#1}}}
\newcommand{\NormalTok}[1]{#1}
\newcommand{\OperatorTok}[1]{\textcolor[rgb]{0.81,0.36,0.00}{\textbf{#1}}}
\newcommand{\OtherTok}[1]{\textcolor[rgb]{0.56,0.35,0.01}{#1}}
\newcommand{\PreprocessorTok}[1]{\textcolor[rgb]{0.56,0.35,0.01}{\textit{#1}}}
\newcommand{\RegionMarkerTok}[1]{#1}
\newcommand{\SpecialCharTok}[1]{\textcolor[rgb]{0.00,0.00,0.00}{#1}}
\newcommand{\SpecialStringTok}[1]{\textcolor[rgb]{0.31,0.60,0.02}{#1}}
\newcommand{\StringTok}[1]{\textcolor[rgb]{0.31,0.60,0.02}{#1}}
\newcommand{\VariableTok}[1]{\textcolor[rgb]{0.00,0.00,0.00}{#1}}
\newcommand{\VerbatimStringTok}[1]{\textcolor[rgb]{0.31,0.60,0.02}{#1}}
\newcommand{\WarningTok}[1]{\textcolor[rgb]{0.56,0.35,0.01}{\textbf{\textit{#1}}}}
\usepackage{longtable,booktabs,array}
\usepackage{calc} % for calculating minipage widths
% Correct order of tables after \paragraph or \subparagraph
\usepackage{etoolbox}
\makeatletter
\patchcmd\longtable{\par}{\if@noskipsec\mbox{}\fi\par}{}{}
\makeatother
% Allow footnotes in longtable head/foot
\IfFileExists{footnotehyper.sty}{\usepackage{footnotehyper}}{\usepackage{footnote}}
\makesavenoteenv{longtable}
\usepackage{graphicx}
\makeatletter
\def\maxwidth{\ifdim\Gin@nat@width>\linewidth\linewidth\else\Gin@nat@width\fi}
\def\maxheight{\ifdim\Gin@nat@height>\textheight\textheight\else\Gin@nat@height\fi}
\makeatother
% Scale images if necessary, so that they will not overflow the page
% margins by default, and it is still possible to overwrite the defaults
% using explicit options in \includegraphics[width, height, ...]{}
\setkeys{Gin}{width=\maxwidth,height=\maxheight,keepaspectratio}
% Set default figure placement to htbp
\makeatletter
\def\fps@figure{htbp}
\makeatother
\setlength{\emergencystretch}{3em} % prevent overfull lines
\providecommand{\tightlist}{%
  \setlength{\itemsep}{0pt}\setlength{\parskip}{0pt}}
\setcounter{secnumdepth}{-\maxdimen} % remove section numbering
\ifluatex
  \usepackage{selnolig}  % disable illegal ligatures
\fi

\title{NYPD}
\author{J. Baldrich}
\date{6/8/2021}

\begin{document}
\maketitle

\hypertarget{nypd-shooting-incident-data-historic}{%
\subsection{NYPD Shooting Incident Data
(Historic)}\label{nypd-shooting-incident-data-historic}}

This dataset includes all valid felony, misdemeanor, and violation
crimes reported to the New York City Police Department (NYPD) from 2006
to the end of 2019.

\hypertarget{nypd-complaints-incident-level-data-footnotes}{%
\subsubsection{NYPD Complaints Incident Level Data
Footnotes}\label{nypd-complaints-incident-level-data-footnotes}}

\begin{enumerate}
\def\labelenumi{\arabic{enumi}.}
\tightlist
\item
  Information is accurate as of the date it was queried from the system
  of record, but should be considered a close approximation of current
  records, due to complaint revisions and updates.
\item
  Data is available as of the date that technological enhancements to
  information systems allowed for data capture. Null values appearing
  frequently in certain fields may be attributed to changes on official
  department forms where data was previously not collected. Null values
  may also appear in instances where information was not available or
  unknown at the time of the report and should be considered as either
  ``Unknown/Not Available/Not Reported.''
\item
  Crime complaints which involve multiple offenses are classified
  according to the most serious offense. Attempted crimes are recorded
  regardless of whether or not the criminal act was successful, except
  in the instance of Attempted Murder, which is recorded as Felony
  Assault.
\item
  When a complaint contains only a From Datetime, this represents the
  exact datetime when the crime incident was reported to occur. In the
  event a complaint has both a From Datetime and a To Datetime, a time
  range (rather than an exact time) was specified for the occurrence of
  the crime. In rare cases, records containing only a To Datetime
  indicate only a known endpoint to the crime.
\item
  The Report Date represents the date the incident was actually reported
  to the NYPD. This dataset is released based on the date the incident
  was reported, not necessarily when it occurred. Some crimes may have
  occurred years before they were reported.
\item
  There is one exception to the above report date rule: if an incident
  eventually results in a victim's death, the incident is upgraded to a
  murder and the report date is recorded as the date of the victim's
  death, rather than the original report date of the incident.
\item
  The Complaint Time is based on the 24-hour clock. 00:01:00 hours
  represents 12:01am, 13:59:00 hours represents 1:59pm, etc.
\item
  Arrests occurring near an intersection are represented by the X
  coordinate and Y coordinates of the intersection. Arrests occurring
  anywhere other than at an intersection are geo-located to the middle
  of the nearest street segment where appropriate.
\item
  Any attempt to match the approximate location of the incident to an
  exact address or link to other datasets is not recommended.
\item
  To further protect victim identities, rape and sex crime offenses have
  been located as occurring at the police station house within the
  precinct of occurrence.
\item
  Many other offenses that were not able to be geo-coded (for example,
  due to an invalid address) have been located as occurring at the
  police station house within the precinct of occurrence.
\item
  Offenses occurring in open areas such as parks or beaches may be
  geo-coded as occurring on streets or intersections bordering the area.
\item
  Offenses occurring on a moving train on transit systems are geo-coded
  as occurring at the train's next stop.
\item
  The NYC subway system is divided by transit district boundaries which
  behave similar to precinct boundaries. Staten Island Rapid Transit
  does not fall under the NYPD transit jurisdiction but instead falls
  under MTA jurisdiction.
\item
  All offenses occurring within the jurisdiction of the Department of
  Correction have been geo-coded as occurring on Riker's Island.
\item
  X and Y Coordinates are in NAD 1983 State Plane New York Long Island
  Zone Feet (EPSG 2263).
\item
  Latitude and Longitude Coordinates are provided in Global Coordinate
  System WGS 1984 decimal degrees (EPSG 4326).
\item
  These data represent criminal offenses according to New York State
  Penal Law definitions, not FBI Uniform Crime Report definitions, and
  are therefore not comparable to UCR reported crime.
\item
  Errors in data transcription may result in nominal data
  inconsistencies.
\item
  The CSV file should be opened using an appropriate tool for data
  exploration, e.g.~SPSS, SAS, Tableau, etc. If using MS Excel, be sure
  to use the tools for importing external data, otherwise
  inconsistencies may occur when viewing the data.
\item
  Only valid complaints are included in this release. Complaints deemed
  unfounded due to reporter error or misinformation are excluded from
  the data set, as they are not reflected in official figures nor are
  they considered to have actually occurred in a criminal context.
  Similarly, complaints that were voided due to internal error are also
  excluded from the data set.
\item
  Investigation reports have been excluded to better ensure relevance
  and reduce extraneous material. These represent complaint reports
  taken that do not indicate or imply that any criminal activity has
  occurred; for example, a natural death of an elderly person in a
  nursing home or a report of lost property that has not been stolen.
\item
  Some \emph{mala prohibita} offenses do not require a complaint report
  and may not be represented accurately, or at all, in this dataset.
  These incidents are generally tracked using other Department forms,
  including arrests and summonses. These include (but are not limited
  to) certain drug, trespassing, theft of service, and prostitution
  offenses.
\item
  Field Names and Descriptions are as follows:
\end{enumerate}

\begin{center}\rule{0.5\linewidth}{0.5pt}\end{center}

\hypertarget{age-of-the-perpetrator}{%
\subsubsection{Age of the perpetrator}\label{age-of-the-perpetrator}}

Out of all the instances in which the age of the perpetrator was
recorded, we can see that people on the younger side tend to appear more
in police reports.

\includegraphics{nypd_files/figure-latex/unnamed-chunk-2-1.pdf}

\hypertarget{time-of-the-crime}{%
\subsubsection{Time of the crime}\label{time-of-the-crime}}

Another insight I thought interesting to plot was the almost evident
idea that crimes seem to be more prevalent at nighttime.

\includegraphics{nypd_files/figure-latex/unnamed-chunk-3-1.pdf}

\hypertarget{time-of-the-crime-age-of-criminal}{%
\subsubsection{Time of the crime \& age of
criminal}\label{time-of-the-crime-age-of-criminal}}

Although not a very useful model, there seems to be a very tenuous
relation between age of the criminal and time of the crime.
Specifically, the later the crime is commited, the younger the criminal
will be.

To calculate this, I converted the text versions of the age brackets
into numbers

\begin{longtable}[]{@{}ll@{}}
\toprule
Text & Number \\
\midrule
\endhead
\textless18 & 16 \\
18-24 & 21 \\
25-44 & 34 \\
45-64 & 54 \\
65+ & 75 \\
\bottomrule
\end{longtable}

\includegraphics{nypd_files/figure-latex/unnamed-chunk-5-1.pdf}

Of course this is very inaccurate, since the ``time of occurrence''
variable does not really explain the age of the criminal very well.

\begin{Shaded}
\begin{Highlighting}[]
\NormalTok{nypd\_both }\SpecialCharTok{\%\textgreater{}\%} \FunctionTok{ggplot}\NormalTok{() }\SpecialCharTok{+} 
  \FunctionTok{geom\_point}\NormalTok{(}\FunctionTok{aes}\NormalTok{(}\AttributeTok{x=}\NormalTok{OCCUR\_TIME, }\AttributeTok{y=}\NormalTok{PERP\_AGE\_GROUP), }\AttributeTok{color =} \StringTok{\textquotesingle{}blue\textquotesingle{}}\NormalTok{) }\SpecialCharTok{+}
  \FunctionTok{geom\_point}\NormalTok{(}\FunctionTok{aes}\NormalTok{(}\AttributeTok{x=}\NormalTok{OCCUR\_TIME, }\AttributeTok{y=}\NormalTok{pred), }\AttributeTok{color =} \StringTok{\textquotesingle{}red\textquotesingle{}}\NormalTok{)}
\end{Highlighting}
\end{Shaded}

\includegraphics{nypd_files/figure-latex/unnamed-chunk-6-1.pdf}

\begin{Shaded}
\begin{Highlighting}[]
\FunctionTok{summary}\NormalTok{(mod)}
\end{Highlighting}
\end{Shaded}

\begin{verbatim}
## 
## Call:
## lm(formula = PERP_AGE_GROUP ~ OCCUR_TIME, data = nypd_both)
## 
## Residuals:
##     Min      1Q  Median      3Q     Max 
## -11.369  -6.250  -5.783   6.974  48.265 
## 
## Coefficients:
##               Estimate Std. Error t value Pr(>|t|)    
## (Intercept)  2.737e+01  1.612e-01 169.806   <2e-16 ***
## OCCUR_TIME  -7.383e-06  2.910e-06  -2.537   0.0112 *  
## ---
## Signif. codes:  0 '***' 0.001 '**' 0.01 '*' 0.05 '.' 0.1 ' ' 1
## 
## Residual standard error: 9.421 on 11948 degrees of freedom
## Multiple R-squared:  0.0005383,  Adjusted R-squared:  0.0004547 
## F-statistic: 6.436 on 1 and 11948 DF,  p-value: 0.0112
\end{verbatim}

\hypertarget{bias}{%
\subsubsection{Bias}\label{bias}}

Since this dataset is a reflection of police reports, it is possible
that it does not translate exactly to real crimes. For instance, it
might be the case that younger people look sketchier, leading people to
report them more often and thus over-representing them in the dataset
used here as a source.

\end{document}
